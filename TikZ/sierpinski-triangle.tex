% Треугольник Серпинского
% Сгенерировано автоматически
\documentclass[tikz,border=10pt]{standalone}
\usepackage[utf8]{inputenc}
\usepackage[russian]{babel}
\usetikzlibrary{lindenmayersystems}
\usetikzlibrary{shadings}

% Определение L-системы для треугольника Серпинского
\pgfdeclarelindenmayersystem{Sierpinski triangle}{
  \rule{F -> G-F-G}
  \rule{G -> F+G+F}}

\begin{document}
\begin{tikzpicture}
  % Треугольник Серпинского с градиентом
  \shadedraw [top color=white, bottom color=orange!80, draw=orange!80!black]
  [l-system={Sierpinski triangle, step=2pt, angle=60, axiom=F, order=8}]
  lindenmayer system -- cycle;
  
  % Заголовок
  \node[above, font=\Large\bfseries, text=orange!50!black] at (current bounding box.north)
    {Треугольник Серпинского};
  \node[below, font=\small] at (current bounding box.south)
    {Порядок рекурсии: 8};
\end{tikzpicture}
\end{document}